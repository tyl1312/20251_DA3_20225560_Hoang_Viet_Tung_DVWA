\subsection{Overview}
Authorization bypass is a type of security vulnerability where an attacker gains access to resources, functionality, or data they should not be authorized to access. It happens when an application fails to properly enforce access controls on user actions or objects.

\noindent \textbf{Objective:} View Authorization Bypass page as user gordonb/abc123

\newpage
\subsection{Low level}
\begin{lstlisting} 
<?php
/*

Nothing to see here for this vulnerability, have a look
instead at the dvwaHtmlEcho function in:

* dvwa/includes/dvwaPage.inc.php

*/

?>
\end{lstlisting}
\noindent This source file does not contain any application logic related to the vulnerability itself. Instead, it explicitly indicates that the relevant behavior is implemented elsewhere, specifically in the dvwaHtmlEcho function located in dvwa/includes/dvwaPage.inc.php.

\begin{lstlisting}
  function dvwaHtmlEcho( $pPage ) {
	$menuBlocks = array();

	$menuBlocks[ 'home' ] = array();
	if( dvwaIsLoggedIn() ) {
		$menuBlocks[ 'home' ][] = array( 'id' => 'home', 'name' => 'Home', 'url' => '.' );
		$menuBlocks[ 'home' ][] = array( 'id' => 'instructions', 'name' => 'Instructions', 'url' => 'instructions.php' );
		$menuBlocks[ 'home' ][] = array( 'id' => 'setup', 'name' => 'Setup / Reset DB', 'url' => 'setup.php' );
	}
	else {
		$menuBlocks[ 'home' ][] = array( 'id' => 'setup', 'name' => 'Setup DVWA', 'url' => 'setup.php' );
		$menuBlocks[ 'home' ][] = array( 'id' => 'instructions', 'name' => 'Instructions', 'url' => 'instructions.php' );
	}

	if( dvwaIsLoggedIn() ) {
		$menuBlocks[ 'vulnerabilities' ] = array();
		$menuBlocks[ 'vulnerabilities' ][] = array( 'id' => 'brute', 'name' => 'Brute Force', 'url' => 'vulnerabilities/brute/' );
		$menuBlocks[ 'vulnerabilities' ][] = array( 'id' => 'exec', 'name' => 'Command Injection', 'url' => 'vulnerabilities/exec/' );
		$menuBlocks[ 'vulnerabilities' ][] = array( 'id' => 'csrf', 'name' => 'CSRF', 'url' => 'vulnerabilities/csrf/' );
		$menuBlocks[ 'vulnerabilities' ][] = array( 'id' => 'fi', 'name' => 'File Inclusion', 'url' => 'vulnerabilities/fi/.?page=include.php' );
		$menuBlocks[ 'vulnerabilities' ][] = array( 'id' => 'upload', 'name' => 'File Upload', 'url' => 'vulnerabilities/upload/' );
		$menuBlocks[ 'vulnerabilities' ][] = array( 'id' => 'captcha', 'name' => 'Insecure CAPTCHA', 'url' => 'vulnerabilities/captcha/' );
		$menuBlocks[ 'vulnerabilities' ][] = array( 'id' => 'sqli', 'name' => 'SQL Injection', 'url' => 'vulnerabilities/sqli/' );
		$menuBlocks[ 'vulnerabilities' ][] = array( 'id' => 'sqli_blind', 'name' => 'SQL Injection (Blind)', 'url' => 'vulnerabilities/sqli_blind/' );
		$menuBlocks[ 'vulnerabilities' ][] = array( 'id' => 'weak_id', 'name' => 'Weak Session IDs', 'url' => 'vulnerabilities/weak_id/' );
		$menuBlocks[ 'vulnerabilities' ][] = array( 'id' => 'xss_d', 'name' => 'XSS (DOM)', 'url' => 'vulnerabilities/xss_d/' );
		$menuBlocks[ 'vulnerabilities' ][] = array( 'id' => 'xss_r', 'name' => 'XSS (Reflected)', 'url' => 'vulnerabilities/xss_r/' );
		$menuBlocks[ 'vulnerabilities' ][] = array( 'id' => 'xss_s', 'name' => 'XSS (Stored)', 'url' => 'vulnerabilities/xss_s/' );
		$menuBlocks[ 'vulnerabilities' ][] = array( 'id' => 'csp', 'name' => 'CSP Bypass', 'url' => 'vulnerabilities/csp/' );
		$menuBlocks[ 'vulnerabilities' ][] = array( 'id' => 'javascript', 'name' => 'JavaScript Attacks', 'url' => 'vulnerabilities/javascript/' );
		if (dvwaCurrentUser() == "admin") {
			$menuBlocks[ 'vulnerabilities' ][] = array( 'id' => 'authbypass', 'name' => 'Authorisation Bypass', 'url' => 'vulnerabilities/authbypass/' );
		}
		$menuBlocks[ 'vulnerabilities' ][] = array( 'id' => 'open_redirect', 'name' => 'Open HTTP Redirect', 'url' => 'vulnerabilities/open_redirect/' );
		$menuBlocks[ 'vulnerabilities' ][] = array( 'id' => 'encryption', 'name' => 'Cryptography', 'url' => 'vulnerabilities/cryptography/' );
		$menuBlocks[ 'vulnerabilities' ][] = array( 'id' => 'api', 'name' => 'API', 'url' => 'vulnerabilities/api/' );
		# $menuBlocks[ 'vulnerabilities' ][] = array( 'id' => 'bac', 'name' => 'Broken Access Control', 'url' => 'vulnerabilities/bac/' );
	}

	$menuBlocks[ 'meta' ] = array();
	if( dvwaIsLoggedIn() ) {
		$menuBlocks[ 'meta' ][] = array( 'id' => 'security', 'name' => 'DVWA Security', 'url' => 'security.php' );
		$menuBlocks[ 'meta' ][] = array( 'id' => 'phpinfo', 'name' => 'PHP Info', 'url' => 'phpinfo.php' );
	}
	$menuBlocks[ 'meta' ][] = array( 'id' => 'about', 'name' => 'About', 'url' => 'about.php' );

	if( dvwaIsLoggedIn() ) {
		$menuBlocks[ 'logout' ] = array();
		$menuBlocks[ 'logout' ][] = array( 'id' => 'logout', 'name' => 'Logout', 'url' => 'logout.php' );
	}

	$menuHtml = '';

	foreach( $menuBlocks as $menuBlock ) {
		$menuBlockHtml = '';
		foreach( $menuBlock as $menuItem ) {
			$selectedClass = ( $menuItem[ 'id' ] == $pPage[ 'page_id' ] ) ? 'selected' : '';
			$fixedUrl = DVWA_WEB_PAGE_TO_ROOT.$menuItem[ 'url' ];
			$menuBlockHtml .= "<li class=\"{$selectedClass}\"><a href=\"{$fixedUrl}\">{$menuItem[ 'name' ]}</a></li>\n";
		}
		$menuHtml .= "<ul class=\"menuBlocks\">{$menuBlockHtml}</ul>";
	}

	// Get security cookie --
	$securityLevelHtml = '';
	switch( dvwaSecurityLevelGet() ) {
		case 'low':
			$securityLevelHtml = 'low';
			break;
		case 'medium':
			$securityLevelHtml = 'medium';
			break;
		case 'high':
			$securityLevelHtml = 'high';
			break;
		default:
			$securityLevelHtml = 'impossible';
			break;
	}
	// -- END (security cookie)

	$userInfoHtml = '<em>Username:</em> ' . ( dvwaCurrentUser() );
	$securityLevelHtml = "<em>Security Level:</em> {$securityLevelHtml}";
	$securityLevelHtml = "<em>Security Level:</em> {$securityLevelHtml}";
	$localeHtml = '<em>Locale:</em> ' . ( dvwaLocaleGet() );
	$sqliDbHtml = '<em>SQLi DB:</em> ' . ( dvwaSQLiDBGet() );


	$messagesHtml = messagesPopAllToHtml();
	if( $messagesHtml ) {
		$messagesHtml = "<div class=\"body_padded\">{$messagesHtml}</div>";
	}

	$systemInfoHtml = "";
	if( dvwaIsLoggedIn() )
		$systemInfoHtml = "<div align=\"left\">{$userInfoHtml}<br />{$securityLevelHtml}<br />{$localeHtml}<br />{$sqliDbHtml}</div>";
	if( $pPage[ 'source_button' ] ) {
		$systemInfoHtml = dvwaButtonSourceHtmlGet( $pPage[ 'source_button' ] ) . " $systemInfoHtml";
	}
	if( $pPage[ 'help_button' ] ) {
		$systemInfoHtml = dvwaButtonHelpHtmlGet( $pPage[ 'help_button' ] ) . " $systemInfoHtml";
	}

	// Send Headers + main HTML code
	Header( 'Cache-Control: no-cache, must-revalidate');   // HTTP/1.1
	Header( 'Content-Type: text/html;charset=utf-8' );     // TODO- proper XHTML headers...
	Header( 'Expires: Tue, 23 Jun 2009 12:00:00 GMT' );    // Date in the past

	echo "<!DOCTYPE html>

<html lang=\"en-GB\">

	<head>
		<meta http-equiv=\"Content-Type\" content=\"text/html; charset=UTF-8\" />

		<title>{$pPage[ 'title' ]}</title>

		<link rel=\"stylesheet\" type=\"text/css\" href=\"" . DVWA_WEB_PAGE_TO_ROOT . "dvwa/css/main.css\" />

		<link rel=\"icon\" type=\"\image/ico\" href=\"" . DVWA_WEB_PAGE_TO_ROOT . "favicon.ico\" />

		<script type=\"text/javascript\" src=\"" . DVWA_WEB_PAGE_TO_ROOT . "dvwa/js/dvwaPage.js\"></script>

	</head>

	<body class=\"home " . dvwaThemeGet() . "\">
		<div id=\"container\">

			<div id=\"header\">

				<img src=\"" . DVWA_WEB_PAGE_TO_ROOT . "dvwa/images/logo.png\" alt=\"Damn Vulnerable Web Application\" />
                <a href=\"#\" onclick=\"javascript:toggleTheme();\" class=\"theme-icon\" title=\"Toggle theme between light and dark.\">
                    <img src=\"" . DVWA_WEB_PAGE_TO_ROOT . "dvwa/images/theme-light-dark.png\" alt=\"Damn Vulnerable Web Application\" />
                </a>
			</div>

			<div id=\"main_menu\">

				<div id=\"main_menu_padded\">
				{$menuHtml}
				</div>

			</div>

			<div id=\"main_body\">

				{$pPage[ 'body' ]}
				<br /><br />
				{$messagesHtml}

			</div>

			<div class=\"clear\">
			</div>

			<div id=\"system_info\">
				{$systemInfoHtml}
			</div>

			<div id=\"footer\">

				<p>Damn Vulnerable Web Application (DVWA)</p>
				<script src='" . DVWA_WEB_PAGE_TO_ROOT . "dvwa/js/add_event_listeners.js'></script>

			</div>

		</div>

	</body>

</html>";
}
\end{lstlisting}

\noindent In the provided code, the dvwaHtmlEcho() function dynamically generates the menu based on the user's authentication state and their role. The Authorization Bypass page is only visible to the admin in the UI, but there is no server-side validation to ensure that only admins can access this page. This means that non-admin users like gordonb can still directly access the page by using the following URL "http://192.168.54.6/DVWA/vulnerabilities/authbypass/".

\begin{figure}[H]
  \centering
  \includegraphics[width=0.7\linewidth]{Figure/auth_res_low.png} 
\end{figure}
\noindent Here, we can see that we still have access to the page.

\noindent \textbf{Patching:}
\noindent To fix this issue, we need to ensure that access control is enforced on the server side, not just in the UI. This involves checking the user's role allowing access to the page. If the user is not an admin, they should be denied access with an HTTP 403 (Forbidden) status.
\begin{lstlisting}
<?php
if (dvwaCurrentUser() != "admin") {
    print "Unauthorised";
    http_response_code(403);
    exit;
}
?>

\end{lstlisting}

\newpage
\subsection{Medium level}
\begin{lstlisting}
<?php
/*

Only the admin user is allowed to access this page.

Have a look at these two files for possible vulnerabilities: 

* vulnerabilities/authbypass/get_user_data.php
* vulnerabilities/authbypass/change_user_details.php

*/

if (dvwaCurrentUser() != "admin") {
    print "Unauthorised";
    http_response_code(403);
    exit;
}
?>
\end{lstlisting}
\noindent Looking at the code, we notice that the page is now restricted to only admin user now. Any non-admin user trying to access the web page will get an "Unauthorised" message. However, the application suggests we visit two URL: vulnerabilities/authbypass/get\_user\_data.php and vulnerabilities/authbypass/change\_user\_details.php. After trying to access the former, we find out that now we can view user details but in JSON format. This suggests that user gordonb can still view user data despite not having authorisation
\begin{figure}[H]
  \centering
  \includegraphics[width=0.7\linewidth]{Figure/auth_med2.png} 
\end{figure}

\noindent \textbf{Patching:}
\noindent The patching method used for low level can be applied here but for every endpoint. This helps protect not only the main page but also helper scripts such as get\_user\_data.php and change\_user\_details.php. If these endpoints are intended for admin only, enforce dvwaCurrentUser() == "admin" inside each script. 

\newpage
\subsection{High level}
\begin{lstlisting}
<?php
/*

Only the admin user is allowed to access this page.

Have a look at this file for possible vulnerabilities: 

* vulnerabilities/authbypass/change_user_details.php

*/

if (dvwaCurrentUser() != "admin") {
    print "Unauthorised";
    http_response_code(403);
    exit;
}
?>
\end{lstlisting}
\noindent The source code doesn't change much for this level but when we try to access vulnerabilities/authbypass/get\_user\_data.php, it returns "Access denied" message. 

\begin{figure}[H]
  \centering
  \includegraphics[width=0.7\linewidth]{Figure/auth_high1.png}
\end{figure}
\noindent Moreover, when we try to access vulnerabilities/authbypass/change\_user\_details.php, it show a error message indicating only POST request is allowed.

\begin{figure}[H]
  \centering
  \includegraphics[width=0.7\linewidth]{Figure/auth_high2.png}
\end{figure}
\noindent Therefore, we would change from GET method to POST in Burp Suite. However, now we get another error suggesting invalid format.

\begin{figure}[H]
  \centering
  \includegraphics[width=0.7\linewidth]{Figure/auth_high3.png}
\end{figure}
\noindent As a result, we need to use the correct JSON format that we have seen in vulnerabilities/authbypass/get\_user\_data.php which is {"id":1,"username":"Bob","last\_name":"Bob"} to our POST request and send.
Now, we receive a 200 OK status with message suggesting success.
\begin{figure}[H]
  \centering
  \includegraphics[width=0.7\linewidth]{Figure/auth_high4.png} 
\end{figure}

\noindent \textbf{Patching:}
Like in medium level, we need to enforce access control checks on every related endpoint, including change\_user\_details.php. If these endpoints are intended for admin only, enforce dvwaCurrentUser() == "admin" inside each script.