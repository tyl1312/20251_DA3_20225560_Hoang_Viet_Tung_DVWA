\documentclass[../main.tex]{subfiles}
\begin{document}
Lưu ý: Sinh viên không được đưa bài giảng/slide, các trang Wikipedia, hoặc các trang web thông thường làm tài liệu tham khảo. 

Một trang web được phép dùng làm tài liệu tham khảo \textbf{chỉ khi} nó là công bố chính thống của cá nhân hoặc tổ chức nào đó. Ví dụ, trang web đặc tả ngôn ngữ XML của tổ chức W3C \url{https://www.w3.org/TR/2008/REC-xml-20081126/} là TLTK hợp lệ.

Có năm loại tài liệu tham khảo mà sinh viên phải tuân thủ đúng quy định về cách thức liệt kê thông tin như sau. Lưu ý: các phần văn bản trong cặp dấu < > dưới đây chỉ là hướng dẫn khai báo cho từng loại tài liệu tham khảo; sinh viên cần xóa các phần văn bản này trong ĐATN của mình.

<\textbf{Bài báo đăng trên tạp chí khoa học}: Tên tác giả, tên bài báo, tên tạp chí, volume, từ trang đến trang (nếu có), nhà xuất bản, năm xuất bản >




<\textbf{Tập san Báo cáo Hội nghị Khoa học}: Tên tác giả, tên báo cáo, tên hội nghị, ngày (nếu có), địa điểm hội nghị, năm xuất bản>

<\textbf{Tài liệu tham khảo từ Internet}: Tên tác giả (nếu có), tựa đề, cơ quan (nếu có), địa chỉ trang web, thời gian lần cuối truy cập trang web>


\end{document}