\subsection{Overview}
\noindent Open redirection vulnerabilities arise when an application incorporates user-controllable data into the target of a redirection in an unsafe way. An attacker can construct a URL within the application that causes a redirection to an arbitrary external domain. This behavior can be leveraged to facilitate phishing attacks against users of the application. The ability to use an authentic application URL, targeting the correct domain and with a valid SSL certificate (if SSL is used), lends credibility to the phishing attack because many users, even if they verify these features, will not notice the subsequent redirection to a different domain.

\noindent \textbf{Objective:} Abuse the redirect page to move the user off the DVWA site or onto a different page on the site than expected.

\newpage
\subsection{Low level}
\begin{lstlisting}
<?php
$errors = "";
$success = "";
$messages = "";

if ($_SERVER['REQUEST_METHOD'] == "POST") {
}

echo "
<p>
    Versioning is important in APIs, running multiple versions of an API can allow for backward compatibility and can allow new services to be added without affecting existing users. The downside to keeping old versions alive is when those older versions contain vulnerabilities.
</p>
";

echo "
<script>
    function update_username(user_json) {
        console.log(user_json);
        var user_info = document.getElementById ('user_info');
        var name_input = document.getElementById ('name');

        if (user_json.name == '') {
            user_info.innerHTML = 'User details: unknown user';
            name_input.value = 'unknown';
        } else {
            if (user_json.level == 0) {
                level = 'admin';
            } else {
                level = 'user';
            }
            user_info.innerHTML = 'User details: ' + user_json.name + ' (' + level + ')';
            name_input.value = user_json.name;
        }

        const message_line = document.getElementById ('message');
        if (user_json.id == 2 && user_json.level == 0) {
            message_line.style.display = 'block';
        } else {
            message_line.style.display = 'none';
        }
    }

    function get_users() {
        const url = '/vulnerabilities/api/v2/user/';
         
        fetch(url, { 
                method: 'GET',
            }) 
            .then(response => { 
                if (!response.ok) { 
                    throw new Error('Network response was not ok'); 
            } 
            return response.json(); 
            }) 
            .then(data => { 
                loadTableData(data);
            }) 
            .catch(error => { 
                console.error('There was a problem with your fetch operation:', error); 
        }); 
    }

    HTMLTableRowElement.prototype.insert_th_Cell = function(index) {
        let cell = this.insertCell(index)
        , c_th = document.createElement('th');
        cell.replaceWith(c_th);
        return c_th;
    }

    function loadTableData(items) {
        const table = document.getElementById('table');
        const tableHead = table.createTHead();
        const row = tableHead.insertRow(0);

        item = items[0];
        Object.keys(item).forEach(function(k){
            let cell = row.insert_th_Cell(-1);
            cell.innerHTML = k;
            if (k == 'password') {
                successDiv = document.getElementById ('message');
                successDiv.style.display = 'block';
            }
        });

        const tableBody = document.getElementById('tableBody');

        items.forEach( item => {
            let row = tableBody.insertRow();
            for (const [key, value] of Object.entries(item)) {
                let cell = row.insertCell(-1);
                cell.innerHTML = value;
            }
        });
    }
    </script>
";

echo "

<table id='table' class=''>
  <thead>
    <tr id='tableHead'>
    </tr>
  </thead>
  <tbody id='tableBody'></tbody>
</table>


        <p>
            Look at the call used to create this table and see if you can exploit it to return some additional information.
        </p>
        <div class='success' style='display:none' id='message'>Well done, you found the password hashes.</div>
        <script>
            get_users();
        </script>
";

?>
\end{lstlisting}
\noindent The source code checks if redirect parameter exists in the URL. If yes, it redirects the user, otherwise, it sets HTTP status code to 500 Internal Server Error. When we click "Quote 1" and take a look at Burp Suite to see what the request looks like,

\begin{figure}[H]
  \centering
  \includegraphics[width=0.7\linewidth]{Figure/http1.png} 
\end{figure}
\noindent Here, we notice that we are visiting /open\_redirect/source/low.php where redirect parameter is "info.php?id=1". Since there is no validation in user input, we can try changing the value of redirect of parameter to a different site such as "https://www.google.com"

\begin{figure}[H]
  \centering
  \includegraphics[width=0.7\linewidth]{Figure/http_low1.png} 
\end{figure}
\noindent We get status 302 Found suggesting that it has been successfully redirected.

\newpage
\subsection{Medium level}
\begin{lstlisting}
<?php

echo "
    <script>
        function update_username(user_json) {
            console.log(user_json);
            var user_info = document.getElementById ('user_info');
            var name_input = document.getElementById ('name');

            if (user_json.name == '') {
                user_info.innerHTML = 'User details: unknown user';
                name_input.value = 'unknown';
            } else {
                var level = 'unknown';
                if (user_json.level == 0) {
                    level = 'admin';
                    successDiv = document.getElementById ('message');
                    successDiv.style.display = 'block';
                } else {
                    level = 'user';
                }
                user_info.innerHTML = 'User details: ' + user_json.name + ' (' + level + ')';
                name_input.value = user_json.name;
            }
        }

        function get_user() {
            const url = '/vulnerabilities/api/v2/user/2';
             
            fetch(url, { 
                    method: 'GET',
                }) 
                .then(response => { 
                    if (!response.ok) { 
                        throw new Error('Network response was not ok'); 
                } 
                return response.json(); 
                }) 
                .then(data => { 
                    update_username (data);
                }) 
                .catch(error => { 
                    console.error('There was a problem with your fetch operation:', error); 
            }); 
        }

        function update_name() {
            const url = '/vulnerabilities/api/v2/user/2';
            const name = document.getElementById ('name').value;
            const data = JSON.stringify({name: name});
             
            fetch(url, { 
                    method: 'PUT', 
                    headers: { 
                        'Content-Type': 'application/json' 
                    }, 
                    body: data
                }) 
                .then(response => { 
                    if (!response.ok) { 
                        throw new Error('Network response was not ok'); 
                } 
                return response.json(); 
                }) 
                .then(data => { 
                    update_username(data);
                }) 
                .catch(error => { 
                    console.error('There was a problem with your fetch operation:', error); 
            }); 
        }
    </script>
";

echo "
        <p>
            Look at the call used to update your name and exploit it to elevate your user to admin (level 0).
        </p>
        <p id='user_info'></p>
        <form method='post' action=\"" . $_SERVER['PHP_SELF'] . "\">
            <p>
                <label for='name'>Name</label>
                <input type='text' value='' name='name' id='name'>
            </p>
            <p>
                <input type=\"button\" value=\"Submit\" onclick='update_name();'>
            </p>
        </form>
        <div class='success' style='display:none' id='message'>Well done, you elevated your user to admin.</div>
        <script>
            get_user();
        </script>
";

?>
\end{lstlisting}
\noindent In the medium level, there is a blacklist to block all URL having https:// or http:// suggesting that no absolute URL is allowed. However, we can use relative URL (just "google.com) which does not contain http:// or https:// protocol, hence bypassing the security check.

\begin{figure}[H]
  \centering
  \includegraphics[width=0.7\linewidth]{Figure/http_med1.png} 
\end{figure}
\noindent Here, we also get status 302 Found suggesting successful redirection.

\newpage
\subsection{High level}
\begin{lstlisting}
<?php

$message = "";

echo "
    <p>
        Here is the <a href='openapi.yml'>OpenAPI</a> document, have a look the health functions and see if you can find one that has a vulnerability.
    </p>
    <p>
        You might be able to work out how to call the individual functions by hand, but it would be a lot easier to import it into an application such as <a href='https://swagger.io/tools/swagger-ui/'>Swagger UI</a>, <a href='https://portswigger.net/bappstore/6bf7574b632847faaaa4eb5e42f1757c'>Burp</a>, <a href='https://www.zaproxy.org/docs/desktop/addons/openapi-support/'>ZAP</a>, or <a href='https://www.postman.com/'>Postman</a> and let the tool do the hard work of setting the requests up for you.
    </p>
";

?>

\end{lstlisting}
\noindent Now, the application sets even stricter rule that we can only redirect to info page at info.php. However, it seems like that the application only checks whether "info.php" exists in redirect value, therefore, we can bypass it by providing a redirect value containing "info.php" such as "google.com?id=info.php"

\begin{figure}[H]
  \centering
  \includegraphics[width=0.7\linewidth]{Figure/http_high1.png}
\end{figure}
\noindent Status 302 Found suggests that our redirect value has worked.