\documentclass[../main.tex]{subfiles}
\begin{document}

\section{Problem Statement}
Web applications are widely used in areas such as online banking, shopping, communication, and data management. As their usage increases, security challenges have also become more serious. Many web applications still contain common weaknesses such as SQL injection, cross-site scripting (XSS), brute force attacks, file inclusion, and command injection. These vulnerabilities allow attackers to steal information, change system data, or take control of servers.

Although there are tools and documents for learning web application security, they are often incomplete or not well connected. Some resources focus only on attack techniques, while others explain only defense strategies. Because of this separation, learners may find it difficult to fully understand how vulnerabilities are discovered, exploited, and prevented in real environments.

To address this gap, this research studies the Damn Vulnerable Web Application (DVWA), a purposely insecure testing platform. The research analyzes seventeen types of web vulnerabilities across three difficulty levels, examining how each attack works, identifying weaknesses in the source code, and demonstrating exploitation methods. In addition, this work applies Web Application Firewall (WAF) rules using ModSecurity to defend against major vulnerabilities. By combining both attack and defense perspectives, the research provides a more complete understanding of web application security.

\section{Contributions}
This research makes the following contributions:

\begin{enumerate}
    \item Provides a detailed analysis of seventeen common web application vulnerabilities in DVWA across various security levels.
    \item Explains how each vulnerability works, including detailed demonstrations of attack methods and code analysis.
    \item Implements ModSecurity WAF rules to defend against major attacks such as brute force, command injection, file inclusion, file upload, SQL injection, and XSS.
    \item Demonstrates practical results showing how WAF rules detect and block real attack attempts.
    \item Discusses the limitations of WAF protection and identifies vulnerabilities requiring application-level fixes.
\end{enumerate}

\section{Organization of research}

The remainder of this research is organized as follows:

Chapter 2 presents a detailed analysis of seventeen vulnerabilities contained in the Damn Vulnerable Web Application DVWA. For each vulnerability, the chapter examines the application logic and source code, demonstrates exploitation methods and discusses corresponding mitigation techniques and patching recommendations at all three levels.

Chapter 3 introduces ModSecurity and the concept of Web Application Firewalls. This chapter explains the process of designing custom WAF rules based on observed vulnerabilities and implements defensive rules for selected attacks including brute force, command injection, file inclusion, file upload exploitation, SQL injection, and cross-site scripting. Practical experiments are also presented to show how these rules detect and block real attack attempts.

Chapter 4 concludes the research. It summarizes the main findings obtained from the vulnerability assessments and WAF implementations, evaluates the effectiveness and limitations of ModSecurity-based protection, and identifies areas for further work and improvement in web application security.
\end{document}